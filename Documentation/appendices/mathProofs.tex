\chapter{Derivations}

\section{$\ln(-1/z)-\ln(1/z)=\sgn\big[\im(z)\big]i\pi$}
\label{s:ln1z}

This section proves that for some complex number $z = a + bi$ with $b \neq 0$, then
\begin{equation}
	\ln\Big(-\frac{1}{z}\Big) - \ln\Big(\frac{1}{z}\Big) = \sgn[\im(z)\big] i \pi.
	\label{eq:ln1z}
\end{equation}
In general, we can use log laws to show that this is somewhat the case.
\begin{align}
	\ln\Big(-\frac{1}{z}\Big) - \ln\Big(\frac{1}{z}\Big) 
	&= \ln(-1) - \ln(z) - \big[ \ln(1) - \ln(z) \big] \nonumber \\
	&= i \pi - \ln(z) - 0 + \ln(z) \nonumber \\
	&= i \pi
\end{align}
This is slightly different than Eq~.\ref{eq:ln1z} by a 
factor of $\sgn(b)$. 
Generally, this should be fine because complex logarithms are multi-valued functions
that are defined by the primary choice of angle plus $2 n \pi$ where $n$ is an arbitrary integer.
However, when doing the actual calculation (with actual numbers in Python or Mathematica), 
we find that the primary choice of angle is dependent on the sign of $b$.
Thus, we will prove this dependency here and thus simplify our solutions significantly.

Consider a complex number $z=a+bi$ with $b \neq 0$. 
We can rewrite this in exponential form as $z = r_0 \exp(i \theta_0)$.
We can obtain the angle $\theta_0$ by obtaining the argument of the complex number, 
or in others taking the inverse tangent.
However, we want the range of the inverse tangent to be in $(-\pi,\pi]$ instead
of the usual $(-\pi/2,\pi/2)$.
Therefore, we will use the $\atan2$ function, or the two argument inverse tangent function
that includes information about the quadrant of the complex number.
Thus, the angle is defined fully as the following depending on several characteristics of $a$ and $b$.
Table~\ref{t:theta0} shows how to obtain $\theta_0$ based on where $z$ is on the complex plane.
We can also get what $\tan^{-1}(b/a)$ is in terms of $\theta_0$, which will be important for 
proving Eq.~\ref{eq:ln1z}.
\begin{table}[H]
	\centering
	\caption{Obtaining $\theta_0$ based on $\atan2$ function depending on location of $z$ in the complex plane.
		Then, we obtain $\tan^{-1}(b/a)$ in terms of $\theta_0$, which is used for future calculations.
		Note we assume that $b\neq0$.}
		\label{t:theta0}
	\begin{tabular}{c|c|c|c}
	\makecell[t]{\textbf{Quadrant \RN{2}}\\
		$a<0\ \&\ b > 0$ \\
		$\theta_0 = \tan^{-1} \big( \frac{b}{a} \big) + \pi$ \\
		$\tan^{-1} \big( \frac{b}{a} \big) = \pi - \theta_0$  \vspace{2pt}}
	&
	\makecell[t]{\textbf{Positive Complex Axis} \\
		$a=0\ \&\ b > 0$ \\
		$\theta_0 = \frac{\pi}{2}$ }
	&
	\makecell[t]{\textbf{Quadrant \RN{1}}\\
		$a>0\ \&\ b > 0$ \\
		$\theta_0 = \tan^{-1} \big( \frac{b}{a} \big)$ \\}
	&
	\makecell[t]{\textbf{Positive Real Axis}\\
	$a> 0\ \&\ b = 0$ \\
	$\theta_0 = 0$
	}
	\\
	\hline
	\makecell[t]{\textbf{Quadrant \RN{3}}\\
		$a<0\ \&\ b < 0$ \\
		$\theta_0 = \tan^{-1} \big( \frac{b}{a} \big) - \pi$ \\
		$\tan^{-1} \big( \frac{b}{a} \big) = -\pi - \theta_0$  }
	&
	\makecell[t]{\textbf{Negative Complex Axis} \\
		$a=0\ \&\ b < 0$ \\
		$\theta_0 = -\frac{\pi}{2}$ }
	& 
	\makecell[t]{\textbf{Quadrant \RN{4}}\\
		$a> 0 \ \&\ b < 0$ \\
		$\theta_0 = \tan^{-1} \big( \frac{b}{a} \big)$ \\} 
	&
	\makecell[t]{\textbf{Negative Real Axis}\\
		$a < 0\ \&\ b = 0$ \\
		$\theta_0 = \pi$
	}
	\end{tabular}
\end{table}
It is not $z$ that shows up in Eq.~\ref{eq:ln1z} but $1/z$ and $-1/z$. 
So let us determine what these are. 
The reciprocal of $z$ is
\begin{equation}
	\frac{1}{z} =  \frac{1}{a+bi}.
\end{equation}
To get to an easy to use form, multiply the top and bottom by the complex conjugate of $z$
and then define $1/z$ as
\begin{equation}
	\frac{1}{z} = \frac{a-bi}{a^2 + b^2} = r_1 \exp(i \theta_1).
\end{equation}
Note how the angle of the reciprocal of $z$ is the same as the angle of the complex conjugate of $z$.
This is effectively saying that the quadrant that $1/z$ is in is flipped up/down.
Then, we can calculate $\theta_1$ in a similar way to how we did for $\theta_0$.
In addition, we can relate $\theta_1$ to $\theta_0$ through $\tan^{-1}(b/a)$ because the inverse tangent is an odd function.
Table~\ref{t:theta1} is a similar table to Table~\ref{t:theta0} and is based on where $z$ (note, not $1/z$) is in the complex plane.
\begin{table}[H]
	\centering
	\caption{Obtaining $\theta_1$ for $1/z$ based on $\atan2$ function depending on location of $z$ in the complex plane.
		Then we relate it back to $\theta_0$ through $\tan^{-1}(b/a)$ using the relationships from Table~\ref{t:theta0}.}
	\label{t:theta1}
	\begin{tabular}{c|c|c|c}
		\makecell[t]{\textbf{Quadrant \RN{2}}\\
			$a<0\ \&\ b > 0$ \\
			$1/z$ in Quadrant \RN{3} \\
			$\begin{aligned}
				\theta_1 
				&= \tan^{-1}\big(\tfrac{-b}{a} \big) - \pi \\
				&= - \tan^{-1}\big(\tfrac{b}{a} \big) - \pi \\
				&= -(\pi - \theta_0) - \pi \\
				&= \theta_0 - 2 \pi
			\end{aligned} $ }
		&
		\makecell[t]{\textbf{Positive Complex Axis} \\
			$a=0\ \&\ b > 0$ \\
			$1/z$ along negative complex axis \\
			$\theta_1 = -\frac{\pi}{2}$ }
		&
		\makecell[t]{\textbf{Quadrant \RN{1}}\\
			$a>0\ \&\ b > 0$ \\
			$1/z$ in Quadrant \RN{4} \\
			$\begin{aligned}
				\theta_1 
				&= \tan^{-1}\big(\tfrac{-b}{a} \big) \\
				&= -\tan^{-1}\big(\tfrac{b}{a} \big) \\
				&= -\theta_0
			\end{aligned}$}
		&
		\makecell[t]{\textbf{Positive Real Axis}\\
			$a> 0\ \&\ b = 0$ \\
			$1/z$ along positive real axis\\
			$\theta_1 = 0$
		}
		\\
		\hline
		\makecell[t]{\textbf{Quadrant \RN{3}}\\
			$a<0\ \&\ b < 0$ \\
			$1/z$ in Quadrant \RN{2} \\
			$\begin{aligned}
				\theta_1 
				&= \tan^{-1}\big(\tfrac{-b}{a} \big) + \pi \\
				&= - \tan^{-1}\big(\tfrac{b}{a} \big) + \pi \\
				&= -(-\pi - \theta_0) + \pi \\
				&= \theta_0 + 2 \pi
			\end{aligned} $}
		&
		\makecell[t]{\textbf{Negative Complex Axis} \\
			$a=0\ \&\ b < 0$ \\
			$1/z$ along positive complex axis \\
			$\theta_1 = \frac{\pi}{2}$ }
		& 
		\makecell[t]{\textbf{Quadrant \RN{4}}\\
			$a>0\ \&\ b > 0$ \\
			$1/z$ in Quadrant \RN{1} \\
			$\begin{aligned}
				\theta_1 
				&= \tan^{-1}\big(\tfrac{-b}{a} \big) \\
				&= -\tan^{-1}\big(\tfrac{b}{a} \big) \\
				&= -\theta_0
			\end{aligned}$}
		&
		\makecell[t]{\textbf{Negative Real Axis}\\
			$a< 0\ \&\ b = 0$ \\
			$1/z$ is along negative real axis\\
			$\theta_1 = \pi$
		}
	\end{tabular}
\end{table}
Now we need to do a similar analysis with $-1/z$, which we will define (based on $1/z$) as 
\begin{equation}
	-\frac{1}{z} = \frac{-a + bi}{a^2 + b^2} = r_2 \exp(i \theta_2).
\end{equation}
This is effectively saying that the quadrant that $-1/z$ is in is flipped left/right compared to $z$.
Note that $-1/z$ and $1/z$ should have the same modulus.
Therefore, $r_1 = r_2$.
Table~\ref{t:theta2} shows the same kind of analysis as Table~\ref{t:theta1} but for $\theta_2$.
\begin{table}[H]
	\centering
	\caption{Obtaining $\theta_2$ for $-1/z$ based on $\atan2$ function depending on location of $z$ in the complex plane.
		Then we relate it back to $\theta_0$ through $\tan^{-1}(b/a)$ using the relationships from Table~\ref{t:theta0}.}
	\label{t:theta2}
	\begin{tabular}{c|c|c|c}
		\makecell[t]{\textbf{Quadrant \RN{2}}\\
			$a<0\ \&\ b > 0$ \\
			$-1/z$ in Quadrant \RN{1} \\
			$\begin{aligned}
				\theta_2
				&= \tan^{-1}\big(\tfrac{b}{-a} \big) \\
				&= - \tan^{-1}\big(\tfrac{b}{a} \big) \\
				&= -(\pi - \theta_0)  \\
				&= \theta_0 - \pi
			\end{aligned} $ }
		&
		\makecell[t]{\textbf{Positive Complex Axis} \\
			$a=0\ \&\ b > 0$ \\
			$-1/z$ along positive complex axis \\
			$\theta_2 = \frac{\pi}{2}$ }
		&
		\makecell[t]{\textbf{Quadrant \RN{1}}\\
			$a>0\ \&\ b > 0$ \\
			$-1/z$ in Quadrant \RN{2} \\
			$\begin{aligned}
				\theta_2
				&= \tan^{-1}\big(\tfrac{b}{-a} \big) + \pi\\
				&= -\tan^{-1}\big(\tfrac{b}{a} \big) + \pi\\
				&= -\theta_0 + \pi
			\end{aligned}$}
		&
		\makecell[t]{\textbf{Positive Real Axis}\\
			$a> 0\ \&\ b = 0$ \\
			$-1/z$ along negative real axis\\
			$\theta_2 = \pi$
		}
		\\
		\hline
		\makecell[t]{\textbf{Quadrant \RN{3}}\\
			$a<0\ \&\ b < 0$ \\
			$-1/z$ in Quadrant \RN{4} \\
			$\begin{aligned}
				\theta_2
				&= \tan^{-1}\big(\tfrac{b}{-a} \big) \\
				&= - \tan^{-1}\big(\tfrac{b}{a} \big) \\
				&= -(-\pi - \theta_0)\\
				&= \theta_0 + \pi
			\end{aligned} $}
		&
		\makecell[t]{\textbf{Negative Complex Axis} \\
			$a=0\ \&\ b < 0$ \\
			$-1/z$ along negative complex axis \\
			$\theta_2 = -\frac{\pi}{2}$ }
		& 
		\makecell[t]{\textbf{Quadrant \RN{4}}\\
			$a>0\ \&\ b > 0$ \\
			$-1/z$ in Quadrant \RN{3} \\
			$\begin{aligned}
				\theta_2
				&= \tan^{-1}\big(\tfrac{b}{-a} \big) -\pi \\
				&= -\tan^{-1}\big(\tfrac{b}{a} \big) -\pi\\
				&= -\theta_0 - \pi
			\end{aligned}$}
		&
		\makecell[t]{\textbf{Negative Real Axis}\\
			$a< 0\ \&\ b = 0$ \\
			$-1/z$ along positive real axis\\
			$\theta_2 = 0$
		}
	\end{tabular}
\end{table}

Now, we must evaluate $\ln(-1/z)-\ln(1/z)$. 
We do this using the definition of a logarithm on the complex plane.
For example, if some value $z = r \exp(i \theta)$,
\begin{equation}
	\ln(z) = \ln(r) + i \theta + 2 n \pi,
\end{equation}
where $n$ is an arbitrary integer. 
By convention, the principal value is based on the angle within $(-\pi,\pi]$.
Thus (neglecting the $2n\pi$ term), we can evaluate $\ln(-1/z)-\ln(1/z)$ as
\begin{equation}
	\ln\Big(-\frac{1}{z} \Big) = \ln \Big(\frac{1}{z}\Big) 
	= \ln (r_2) + i \theta_2 - \ln(r_1) - i \theta_1
	= i(\theta_2 - \theta_1).
\end{equation}
Since $r_1=r_2$, these terms cancel, leaving only the angles.
Table~\ref{t:ln} shows the result of this calculation based
on the location of $z$ in the complex plane and Tables~\ref{t:theta1}
and \ref{t:theta2} for $\theta_1$ and $\theta_2$, respectively.
\begin{table}[H]
	\centering
	\caption{Evaluation of $\ln(-1/z)-\ln(1/z)$ based on 
		Tables~\ref{t:theta1} and \ref{t:theta2}.}
	\label{t:ln}
	\begin{tabular}{c|c|c|c}
		\makecell[t]{\textbf{Quadrant \RN{2}}\\
			$a<0\ \&\ b > 0$ \\
			$\begin{aligned}
				i(\theta_2-\theta_1)
				&= i(\theta_0-\pi-\theta_0+2\pi)\\
				&= i\pi 
			\end{aligned} $ }
		&
		\makecell[t]{\textbf{Positive Complex Axis} \\
			$a=0\ \&\ b > 0$ \\
			$\begin{aligned}
				i(\theta_2-\theta_1)
				&= i(\pi/2+\pi/2)\\
				&= i\pi 
			\end{aligned} $ }
		&
		\makecell[t]{\textbf{Quadrant \RN{1}}\\
			$a>0\ \&\ b > 0$ \\
			$\begin{aligned}
				i(\theta_2-\theta_1)
				&= i(-\theta_0+\pi+\theta_0)\\
				&= i\pi 
			\end{aligned} $}
		&
		\makecell[t]{\textbf{Positive Real Axis}\\
			$a> 0\ \&\ b = 0$ \\
			$\begin{aligned}
				i(\theta_2-\theta_1) 
				&= i(\pi-0) \\
				&= i \pi
			\end{aligned}$
		}
		\\
		\hline
		\makecell[t]{\textbf{Quadrant \RN{3}}\\
			$a<0\ \&\ b < 0$ \\
			$\begin{aligned}
				i(\theta_2-\theta_1)
				&= i(\theta_0+\pi-\theta_0-2\pi)\\
				&= -i\pi 
			\end{aligned} $}
		&
		\makecell[t]{\textbf{Negative Complex Axis} \\
			$a=0\ \&\ b < 0$ \\
			$\begin{aligned}
				i(\theta_2-\theta_1)
				&= i(-\pi/2-\pi/2)\\
				&= -i\pi 
			\end{aligned} $ }
		& 
		\makecell[t]{\textbf{Quadrant \RN{4}}\\
			$a>0\ \&\ b > 0$ \\
			$\begin{aligned}
				i(\theta_2-\theta_1)
				&= i(-\theta_0-\pi+\theta_0)\\
				&= -i\pi 
			\end{aligned} $} 
		&
		\makecell[t]{\textbf{Negative Real Axis}\\
			$a< 0\ \&\ b = 0$ \\
			$\begin{aligned}
				i(\theta_2-\theta_1) 
				&= i(0-\pi) \\
				&= -i \pi
			\end{aligned}$}
	\end{tabular}
\end{table}

Based on these tables and the physics of what's going on, 
we find that because $b<0$ always, the solution will always be
$-i\pi$.
%
%What we find is that the top rows, where $b>0$, we get $i \pi$
%and for the bottom rows, where $b<0$, we get $-i\pi$.
%Therefore, we can rewrite this as $\sgn(b)i\pi$.
%The imaginary component of $z$ was defined to be $b$
%showing that Eq.~\ref{eq:ln1z} is correct.
\begin{equation}
	\ln\Big(-\frac{1}{z}\Big) - \ln\Big(\frac{1}{z}\Big) = \sgn[\im(z)\big] i \pi.
\end{equation}



\section{Analytical Solutions for ISR Spectra}
\label{a:spectra}
This section shows how we get into the nice forms of the analytical solutions for the scattering spectra
and its internal terms for a Maxwellian.

As shown in Sec.~\ref{s:ISR-spectra}, the equation for the scattering spectra is
\begin{equation}
	S(\omega,\mathbf{k}) = 2 \Big| 1 - \frac{\chi_e}{\epsilon}\Big|^2
	+ 2\Big|\frac{\chi_e}{\epsilon}\Big|^2 M_i.
\end{equation}
The dielectric function is
\begin{equation}
	\epsilon = 1 + \chi_e + \chi_i.
\end{equation}
We can calculate this using the collisional term, susceptibility, and modified distribution function:
\begin{equation}
	U_s = i\nu_s \sum_n \int
	\frac{J_n^2\Big( \tfrac{k_\perp v_\perp}{\Omega_{cs}} \Big)}
	{\omega-k_\parallel v_\parallel - n\Omega_{cs} - i\nu_s}
	f_{0s}(\mathbf{v}) d\mathbf{v} ,
\end{equation}
\begin{equation}
	\chi_s = \frac{\omega_{ps}^2}{k^2(1+U_s)}
	\sum_n \int
	\frac{J_n^2 \Big( \tfrac{k_\perp v_\perp}{\Omega_{cs}} \Big)}
	{\omega-k_\parallel v_\parallel - n\Omega_{cs} - i\nu_s}
	\mathbf{k} \cdot \frac{\partial f_{0s}}{\partial \mathbf{v}} d\mathbf{v},
\end{equation}
\begin{equation}
	M_s = \frac{\nu_s}{|1+U_s|^2}
	\Bigg( - \frac{|U_s|^2}{\nu_s^2} 
	+ \sum_n \int 
	\frac{J_n^2\Big( \tfrac{k_\perp v_\perp}{\Omega_{cs}} \Big)}
	{(\omega - k_\parallel v_\parallel - n\Omega_{cs})^2 + \nu_s^2}
	f_{0s}(v)   d\mathbf{v}^3 \Bigg),
\end{equation}
where 
\begin{equation}
	\mathbf{k} \cdot \frac{\partial f_{0s}(v) }{\partial \mathbf{v}} = 
		k_\parallel \frac{\partial f_{0s}}{\partial v_\parallel}
		+ \frac{n \Omega_{cs}}{v_\perp} \frac{\partial f_{0s}}{\partial v_\perp}
\end{equation}
We can solve these integrals in cylindrical coordinates using 
\begin{equation}
	\int f_{0s}(\mathbf{v}) d\mathbf{v} = \int_0^{2\pi} \int_0^\infty \int_{-\infty}^\infty v_\perp f_{0s}(\mathbf{v}) 
	dv_\parallel dv_\perp d\phi.
\end{equation}

We can therefore, find the spectra using these equations
for an arbitrary ion distribution function.
However, depending on the shape of the distribution function, there may or may not be an
analytical solution for the scattering spectra.
For the case of a bi-Maxwellian distribution, 
\begin{equation}
	f_{0s} = v_{th,s,\perp}^{-2} v_{th,s,\parallel} \pi^{-3/2} 
	\exp\bigg[ - \frac{v_\perp^2}{v_{th,s,\perp}^2} - \frac{v_\parallel^2}{v_{th,s,\parallel}^2} \bigg],
\end{equation}
we can find an analytical solution.
These integrals are messy to solve so we will use Mathematica (see the notebook \verb|calcS_Maxwellian.nb| for the analytical solutions)  % Update this mathematica notebook

\subsection{Useful Relationships}  % Should have some sort of proof of these somewhere maybe
The following relationships will be useful in simplifying the Mathematica output
\begin{equation}
	y_n = \frac{\omega - n \Omega_{cs}- i \nu_s}{k_\parallel v_{th,s,\parallel}}
	\label{eq:a_yn}
\end{equation}
\begin{equation}
	y_n^* = \frac{\omega - n \Omega_{cs}+ i \nu_s}{k_\parallel v_{th,s,\parallel}}
	\label{eq:a_ynstar}
\end{equation}
\begin{equation}
	\bar{\rho}_s = \frac{v_{th,s,\perp}}{\sqrt{2} \Omega_{cs}}
	\label{eq:a_rho}
\end{equation}
\begin{equation}
	2 \Da(z) = \sqrt{\pi} \exp(-z^2) \erfi(z)
	\label{eq:a_Daw}
\end{equation}
\begin{equation}
	\ln\bigg( -\frac{1}{v_{th,s,\parallel}y_n} \bigg) - \ln\bigg( \frac{1}{v_{th,s,\parallel}y_n} \bigg) = -i \pi
	\label{eq:a_ln}
\end{equation}
\begin{equation}
	k_\perp^2 \bar{\rho}_s^2 = \frac{k_\perp^2 v_{th,s,\perp}^2}{2\Omega_{cs}^2}
	\label{eq:a_k2rho2}
\end{equation}
\begin{equation}
	-y_n = \frac{i\nu + n\Omega{cs}-\omega}{k_\parallel v_{th,s,\parallel}}
	\label{eq:a_-yn}
\end{equation}
\begin{equation}
	-y_n^2 = \frac{\big(\nu-in\Omega_{cs}+i\omega\big)^2}{k_\parallel^2 v_{th,s,\parallel}^2}
	\label{eq:a_-yn2}
\end{equation}
\begin{equation}
	\erfi(-z) = -\erfi(z) 
	\label{eq:a_erfi_odd}
\end{equation}
\begin{equation}
	\frac{1}{v_{th,s}y_n} = \frac{k_\parallel}{-i\nu_s-n\Omega_{cs}+\omega}
	\label{eq:a_vthyn}
\end{equation}
\begin{equation}
	\frac{1}{v_{th,s}y_n^*} = \frac{k_\parallel}{i\nu_s-n\Omega_{cs}+\omega}
	\label{eq:a_vthynstar}
\end{equation}
\begin{equation}
	\frac{\nu_s^2-(n\Omega_{cs}+\omega)^2-2 i \nu_s (n\Omega_{cs}+ \omega)}{k_\parallel^2 v_{th,s,\parallel}^2} = 
	-(y_n^*)^2 - \frac{4 i \nu_s n \Omega_{cs}}{k_\parallel^2 v_{th,s,\parallel}^2}
	= - y_n^2 - \frac{4 i \nu_s \omega}{k_\parallel^2 v_{th,s,\parallel}^2}  
	\label{eq:a_ynstar_plus}
\end{equation}
\begin{equation}
	i \erfi(y_n) = \erf \bigg( \frac{\nu_s - i n\Omega_{cs} + i \omega}{k_\parallel v_{th,s,\parallel}}  \bigg) = \erf(y_n)
	\label{eq:a_erfiyn}
\end{equation}
\begin{equation}
	\alpha = \frac{1}{k \lambda_{D_e}}
	\label{eq:a_alpha}
\end{equation}
\begin{equation}
	\alpha^2 \frac{T_e}{T_s} = \frac{2 \omega_{ps}^2}{k^2 v_{th,s,\parallel}^2}
	\label{eq:a_alpha2}
\end{equation}
\begin{equation}
	-i\pi = \ln \Big[ \frac{k_\parallel}{i\nu_s + n\Omega_{cs} - \omega} \Big]
	- \ln \Big[ \frac{k_\parallel}{-i\nu_s - n\Omega_{cs} + \omega} \Big] 
	\label{eq:a_lnipi}
\end{equation}
\begin{equation}
	i\pi = \ln \Big( -\frac{1}{v_{th,s,\parallel}y_n^*}\Big) - \ln \Big(\frac{1}{v_{th,s,\parallel}y_n^*}\Big)
	\label{eq:a_lnipistar}
\end{equation}
\begin{equation}
	\erfc(x) = 1 - \erf(x)
	\label{eq:a_erfc}
\end{equation}

\subsection{Collisional Term, $U_s$}
The output from Mathematica for the collisional term is
\begin{multline}
	U_s = -\frac{i \nu_s}{k_\parallel \sqrt{\pi}v_{th,s,\parallel}} \sum_n
	\exp \Bigg[ \underbrace{\frac{\big( \nu_s - i n\Omega_{cs} + i \omega \big)^2}{k_\parallel^2 v_{th,s,\parallel}^2}}_{-y_n^2}
	- \underbrace{\frac{k_\perp^2 v_{th,s,\perp}^2}{2 \Omega_{cs}^2}}_{k_\perp^2 \bar{\rho}_s^2} \Bigg]
	I_n \bigg( \underbrace{\frac{k_\perp^2 v_{th,s}^2}{2\Omega_{cs}^2}}_{k_\perp^2 \bar{\rho}_s^2}\bigg)
	\Bigg[\pi \erfi \bigg( \underbrace{\frac{i\nu_s + n\Omega_{cs} - \omega}{k_\parallel v_{th,s}}}_{-y_n}  \bigg)   \\
	+ \ln \bigg( \underbrace{\frac{k_\parallel}{i\nu_s + n\Omega_{cs} - \omega} }_{-\frac{1}{v_{th,s,\parallel}y_n}}\bigg)
	- \ln \bigg( \underbrace{\frac{k_\parallel}{-i\nu_s - n\Omega_{cs} + \omega}}_{-\frac{1}{v_{th,s,\parallel}y_n}} \bigg) 
	\Bigg]
\end{multline}
Using the relationships from Eqs.~\ref{eq:a_k2rho2}, \ref{eq:a_-yn}, \ref{eq:a_-yn2}, and \ref{eq:a_vthyn}, we get
\begin{equation}
	U_s = -\frac{i \nu_s}{k_\parallel \sqrt{\pi} v_{th,s,\parallel}} \sum_n
	\exp \Big(-y_n^2 - k_\perp^2 \bar{\rho}_s^2 \Big)
	 I_n \Big( k_\perp^2 \bar{\rho}_s^2 \Big)	
	\Bigg[ \pi \underbrace{\erfi\Big(-y_n\Big)}_{-\erfi(y_n)}
	+\underbrace{ \ln\bigg( -\frac{1}{v_{th,s,\parallel} y_n} \bigg)
	-\ln\bigg( \frac{1}{v_{th,s,\parallel} y_n} \bigg)}_{-i\pi}
	\Bigg]
\end{equation}
Distribute inside the $-1/\sqrt{\pi}$ and use the relationships
from Eqs.~\ref{eq:a_ln} and \ref{eq:a_erfi_odd} to get
\begin{equation}
	U_s = \frac{i\nu_s}{k_\parallel v_{th,s,\parallel}} \sum_n
	\exp(-y_n^2)
	\exp(-k_\perp^2 \bar{\rho}_s^2)
	 I_n (k_\perp^2 \bar{\rho}_s^2) 
	\Bigg[ - \pi \bigg(-\erfi\Big[y_n\Big]\bigg) - 
	\bigg(- \frac{i\pi}{\sqrt{\pi}} \bigg)
\end{equation}
Distribute the $\exp(-y_n^2)$.
\begin{equation}
	U_s = \frac{i \nu_s}{k_\parallel v_{th,s,\parallel}} \sum_n
	\exp(-k_\perp^2 \bar{\rho}_s^2)
	 I_n (k_\perp^2 \bar{\rho}_s^2)
	\Bigg[ \underbrace{\sqrt{\pi} \exp(-y_n^2) \erfi(y_n)}_{2\Da(y_n)} + i \sqrt{\pi} \exp(-y_n^2) \Bigg]
\end{equation}
Use relationship in Eq.~\ref{eq:a_Daw} to get
\begin{equation}
	U_s = \frac{i \nu_s}{k_\parallel v_{th,s,\parallel}} \sum_n
	\exp(-k_\perp^2 \bar{\rho}_s^2)
	 I_n (k_\perp^2 \bar{\rho}_s^2)
	\bigg[ 2\Da(y_n) + i \sqrt{\pi} \exp(-y_n^2)  \bigg],
\end{equation}
which is the final nice looking result.


\subsection{Susceptibility, $\chi$}
The result from Mathematica is
\begin{multline}
	\chi_s = \frac{2 \omega_{ps}^2}{k_\parallel k^2 \sqrt{\pi} v_{th,s,\parallel}^3 v_{th,s,\perp}^2(1+U_s)}
	\sum_n \exp \Bigg[ \underbrace{\frac{(\nu_s - in\Omega_{cs} + i\omega)^2}{k_\parallel^2 v_{th,s,\parallel}^2}}_{-y_n^2}
	- \underbrace{\frac{k_\perp^2 v_{th,s,\perp}^2}{2 \Omega_{cs}^2}}_{k_\perp^2 \bar{\rho}_s^2}  \Bigg]
	I_n \bigg( \underbrace{\frac{k_\perp^2 v_{th,s,\perp}^2}{2 \Omega_{cs}^2}}_{k_\perp^2 \bar{\rho}_s^2}  \bigg)* \\
	\Bigg[ k_\parallel \sqrt{\pi} v_{th,s,\parallel} v_{th,s,\perp}^2 
	\exp \bigg( - \underbrace{\frac{[\nu_s - i n\Omega_{cs}+ i \omega]^2}{k_\parallel^2 v_{th,s,\parallel}^2}}_{-y_n^2}   \bigg) \\
	+ \bigg( n\Omega_{cs} v_{th,s,\parallel}^2 + \Big[ -i\nu_s - n\Omega_{cs}+ \omega \Big] v_{th,s,\perp}^2  \bigg)
	\bigg( i \pi \underbrace{\erf \Big[ \frac{\nu_s - in\Omega_{cs}+i\omega}{k_\parallel v_{th,s,\parallel}}  \Big]}_{i \erfi(y_n)}\\
	+ \underbrace{\ln \Big[ \frac{k_\parallel}{i\nu_s + n\Omega_{cs} - \omega} \Big]
	- \ln \Big[ \frac{k_\parallel}{-i\nu_s - n\Omega_{cs} + \omega} \Big]}_{-i\pi}
	\bigg)
	\Bigg].
\end{multline}
Use Eqs.~\ref{eq:a_k2rho2}, \ref{eq:a_-yn2}, \ref{eq:a_erfiyn}, and \ref{eq:a_lnipi} and separating the first exponential term, we get
\begin{multline}
	\chi_s = \frac{2\omega_{ps}^2}{k_\parallel k^2 \sqrt{\pi} v_{th,s,\parallel}^3 v_{th,s,\perp}^2(1+U_s)} 
	\sum_n \exp(-k_\perp^2 \bar{\rho}_s^2) \exp(-y_n)^2 I_n(k_\perp^2 \bar{\rho}_s^2)
	\Bigg[ k_\parallel \sqrt{\pi} v_{th,s,\parallel} v_{th,s,\perp}^2 \exp\big( -[y_n^2] \big)  \\+
	\bigg( n\Omega_{cs} v_{th,s,\parallel}^2 + \Big[ -i\nu_s - n\Omega_{cs}+ \omega \Big] v_{th,s,\perp}^2 \bigg)
	\bigg( i \pi i \erfi[y_n] - i\pi \bigg)
	\Bigg].
\end{multline}
Distribute the $1/k_\parallel \sqrt{\pi}$ to get
\begin{multline}
	\chi_s = \frac{2\omega_{ps}^2}{k^2 v_{th,s,\parallel}^3 v_{th,s,\perp}^2 (1+U_s)} \sum_n
	\exp(-k_\perp^2 \bar{\rho}^2) I_n(k_\perp^2 \bar{\rho}^2) \exp(-y_n^2)
	\Bigg[ \frac{k_\parallel \sqrt{\pi} v_{th,s,\parallel} v_{th,s,\perp}^2}{k_\parallel \sqrt{\pi}} \exp(y_n^2) \\
	+ \frac{1}{k_\parallel \sqrt{\pi}} 
	\bigg( n\Omega_{cs} v_{th,s,\parallel}^2 + \Big[ -i\nu_s - n\Omega_{cs}+ \omega \Big] v_{th,s,\perp}^2 \bigg)
	\bigg( - \pi \erfi[y_n] - i\pi \bigg)
	\Bigg].
\end{multline}
Distribute the $k_\parallel$ into the first set of parentheses, the $\sqrt{\pi}$ into the second set, and
the $\exp(-y_n^2)$ into the brackets and factor out a negative one from the parentheses with the $\erfi$ to get
\begin{multline}
	\chi_s = \frac{2\omega_{ps}^2}{k^2 v_{th,s,\parallel}^3v_{th,s,\perp}^2(1+U_s)} \sum_n
	\exp(-k_\perp^2 \bar{\rho}^2) I_n(k_\perp^2 \bar{\rho}^2)
	\Bigg[ v_{th,s,\parallel}v_{th,s,\perp}^2 \exp(y_n^2) \exp(-y_n^2) \\
	- \bigg( \frac{n \Omega_{cs} v_{th,s,\parallel}^2}{k_\parallel} + \Big[ \frac{\omega-n\Omega_{cs}-i\nu_s}{k_\parallel} \Big]v_{th,s,\perp}^2  \bigg)
	\exp(-y_n^2)
	\bigg( \sqrt{\pi} \erfi[y_n] + i \sqrt{\pi}  \bigg)
	\Bigg]
\end{multline}
Distribute $1/v_{th,s,\parallel}v_{th,s,\perp}^2$ from the outside into the brackets and the $\exp(-y_n^2)$
into the parentheses with the $\erfi$ term to get
\begin{multline}
	\chi_s = \frac{1}{1+U_s}\underbrace{\frac{2\omega_{ps}^2}{k^2 v_{th,s,\parallel}^2}}_{\alpha^2 \frac{T_e}{T_s,\parallel}} \sum_n
	\exp(-k_\perp^2 \bar{\rho}^2) I_n(k_\perp^2 \bar{\rho}^2)
	\Bigg[\frac{v_{th,s,\parallel}v_{th,s,\perp}^2}{v_{th,s,\parallel}v_{th,s,\perp}^2} 
	- \frac{1}{v_{th,s,\parallel}v_{th,s,\perp}^2}
	\bigg( \frac{n\Omega_{cs}v_{th,s,\parallel}^2}{k_\parallel} \\+ 
	\Big[ \frac{\omega - n\Omega_{cs} - i \nu_s}{k_\parallel} \Big] v_{th,s,\perp}^2  \bigg)
	\bigg( \underbrace{\sqrt{\pi} \exp[-y_n^2] \erfi[y_n]}_{2\Da(y_n)} + i\sqrt{\pi} \exp[-y_n^2] \bigg)
	\Bigg].
\end{multline}
Using Eqs.~\ref{eq:a_Daw} and \ref{eq:a_alpha2} and distribute the $1/v_{th,s,\parallel}v_{th,s,\perp}^2$ into that first set of parentheses to get
\begin{multline}
	\chi_s = \frac{\alpha^2}{1+U_s} \frac{T_e}{T_{s,\parallel}} \sum_n
	\exp(-k_\perp^2 \bar{\rho}^2) I_n(k_\perp^2 \bar{\rho}^2)
	\Bigg[ 1 - 
	\bigg( \frac{n \Omega_{cs} v_{th,s,\parallel}^2}{k_\parallel v_{th,s,\parallel} v_{th,s,\perp}^2} + 
	\Big[ \underbrace{\frac{\omega - n\Omega_{cs} - i \nu_s}{k_\parallel v_{th,s,\parallel}}}_{y_n} \Big] \frac{v_{th,s,\perp}^2}{v_{th,s,\perp}^2} 
	\bigg) \\
	* \bigg( 2 \Da[y_n] + i \sqrt{\pi} \exp[-y_n]^2
	\bigg)
	\Bigg]
\end{multline}
Use Eq.~\ref{eq:a_yn} and simplify to get
\begin{equation}
	\chi_s = \frac{\alpha^2}{1+U_s} \frac{T_e}{T_{s,\parallel}} \sum_n
	\exp(-k_\perp^2 \bar{\rho}^2) I_n(k_\perp^2 \bar{\rho}^2)
	\Bigg[ 1 - \bigg( \frac{n\Omega_{cs}v_{th,s,\parallel}}{k\parallel v_{th,s,\perp}^2} + y_n \bigg)
	\bigg( 2 \Da[y_n] + i \sqrt{\pi} \exp[-y_n^2] \bigg)
	\Bigg],
\end{equation}
which is the nice final answer.


\subsection{Modified Distribution}
The result from Mathematica is
\begin{multline}
	M_s = \frac{1}{|1+U_s|^2} \Bigg[
	\frac{i}{2k_\parallel \sqrt{\pi} v_{th,s,\parallel}} \sum_n
	\exp\bigg(\underbrace{ \frac{\nu_s^2-[-n\Omega_{cs}+\omega]^2-2i\nu[n\Omega_{cs}+\omega]}{k_\parallel^2 v_{th,s,\parallel}^2}}_{ -(y_n^*)^2 - \frac{4 i \nu_s n \Omega_{cs}}{k_\parallel^2 v_{th,s,\parallel}^2}}
	- \underbrace{\frac{k_\perp^2 v_{th,s,\perp}^2}{2\Omega_{cs}^2}}_{k_\perp^2 \bar{\rho}_s^2} \bigg) \\
	* I_n \bigg( \underbrace{ \frac{k_\perp^2 v_{th,s,\perp}^2}{2\Omega_{cs}^2}}_{k_\perp^2 \bar{\rho}_s^2} \bigg)
	\bigg( \exp\bigg[ \frac{4i\nu_s\omega}{k_\parallel^2v_{th,s,\parallel}^2} \bigg]
	\bigg[ \pi \erfi \bigg( \underbrace{\frac{i\nu_s+n\Omega_{cs}-\omega}{k_\parallel v_{th,s,\parallel}}}_{-y_n} \bigg) 
	+\underbrace{\ln \bigg( \frac{k_\parallel}{i\nu_s + n\Omega_{cs} - \omega} \bigg)
	- \ln \bigg( \frac{k_\parallel}{-i\nu_s - n\Omega_{cs} + \omega} \bigg)}_{-i\pi} 	\bigg] \\
	+ \exp \bigg[ \frac{4i\nu_sn\Omega_{cs}}{k_\parallel^2 v_{th,s,\parallel}^2} \bigg]
	\bigg[ \pi \erfi \bigg( \underbrace{\frac{i\nu_s-n\Omega_{cs}+\omega}{k_\parallel v_{th,s,\parallel}}}_{y_n^*} \bigg) 
	+\ln \bigg( \underbrace{\frac{k_\parallel}{-i\nu_s + n\Omega_{cs} - \omega} }_{-1/v_{th,s,\parallel} y_n^*}\bigg)
	- \ln \bigg( \underbrace{ \frac{k_\parallel}{i\nu_s - n\Omega_{cs} + \omega}}_{1/v_{th,s,\parallel} y_n^*}	\bigg) \Bigg]
	- \frac{|U_s|^2}{\nu_s|1+U_s|^2}.
\end{multline}
Use Eqs.~\ref{eq:a_ynstar}, \ref{eq:a_k2rho2}, \ref{eq:a_-yn}, \ref{eq:a_vthynstar}, \ref{eq:a_ynstar_plus}, and \ref{eq:a_lnipi} to get
\begin{multline}
	M_s = - \frac{|U_s|^2}{\nu_s |1+U_s|^2}+\frac{1}{|1+U_s|^2}
	\Bigg[\frac{i}{2k_\parallel \sqrt{\pi} v_{th,s,\parallel} }\sum_n
	\exp(-k_\perp^2 \bar{\rho}_s^2) I_n (k_\perp^2 \bar{\rho}_s^2)
	\exp\big(-[y_n^*]^2\big)
	\exp \bigg( -\frac{4 i \nu_s n\Omega_{cs}}{k_\parallel^2 v_{th,s,\parallel}^2}\bigg) \\
	* \bigg( \exp \bigg[ \frac{4i\nu_s\omega}{k_\parallel^2 v_{th,s,\parallel}^2}\bigg]
	  \bigg[ \pi \underbrace{\erfi(-y_n)}_{-\erfi(y_n)} - i\pi\bigg]
	  + \exp\bigg[ \frac{4i\nu_sn\Omega_{cs}}{k_\parallel^2v_{th,s,\parallel}^2} \bigg]
	  \bigg[ \pi \erfi(y_n^*)
	   - \ln \bigg( -\frac{1}{v_{th,s,\parallel}y_n^*} \bigg) 
	   + \ln \bigg( \frac{1}{v_{th,s,\parallel}y_n^*} \bigg)\bigg]
	   \bigg)
	\Bigg].
\end{multline}
Use Eq.~\ref{eq:a_erfi_odd}, distribute $\exp(-4i\nu_sn\Omega_{cs}/k_\parallel^2v_{th,s,\parallel}^2)$ into the parentheses,
and factor out a negative one from both of the natural log terms to get
\begin{multline}
	M_s = -\frac{|U_s|^2}{\nu_s|1+U_s|^2} + \frac{1}{|1+U_s|^2} \Bigg[
	\frac{i}{2 k_\parallel \sqrt{\pi} v_{th,s,\parallel}} \sum_n
	\exp\big(-[y_n^*]^2\big) \exp(-k_\perp^2\bar{\rho}_s^2) I_n(k_\perp^2 \bar{\rho}_s^2)\\
	* \bigg( \exp \bigg[ \underbrace{\frac{4i\nu_s\omega}{k_\parallel^2 v_{th,s,\parallel}^2} - 
		\frac{4i\nu_s\Omega_{cs}}{k_\parallel^2v_{th,s,\parallel}^2}}_{\frac{4i\nu_s\omega (\omega - n\Omega_{cs})}{k_\parallel^2v_{th,s,\parallel^2}}} \bigg]
	\bigg[ -\pi \erfi(y_n) - i \pi \bigg]
	+ \underbrace{\exp\bigg[ \frac{4i\nu_sn\Omega_{cs}}{k_\parallel^2 v_{th,s,\parallel}^2} - \frac{4i\nu_s n\Omega_{cs}}{k_\parallel^2 v_{th,s,\parallel}^2} \bigg]}_{1}
	\bigg[ \pi \erfi(y_n^*) \\
	- \bigg( \underbrace{\ln \bigg[ -\frac{1}{v_{th,s,\parallel}y_n^*}\bigg] - \ln \bigg[\frac{1}{v_{th,s,\parallel}y_n^*}\bigg]  \bigg) }_{i\pi}\bigg]
	\bigg)
	\Bigg].
\end{multline}
Combine the terms in the last two exponentials and use Eq.~\ref{eq:a_lnipistar} to get
\begin{multline}
	M_s = -\frac{|U_s|^2}{\nu_s|1+U_s|^2} + \frac{1}{|1+U_s|^2} 
	\Bigg[ \frac{i}{2 k_\parallel \sqrt{\pi} v_{th,s,\parallel}} 
	\sum_n \exp(-k_\perp^2\bar{\rho}^2) I_n(k_\perp^2\bar{\rho}^2)\exp\big( -[y_n^*]^2\big) 	\\
	* \bigg( \exp \bigg[ \frac{i\nu_s (\omega-n\Omega_{cs})}{k_\parallel^2v_{th,s,\parallel}^2} \bigg] 
	\bigg[ \pi \erfi(y_n) - i \pi\bigg] + \pi \erfi(y_n^*) - i \pi \bigg)
	\Bigg].
\end{multline}
Distribute in the $i$ and factor out the $\pi$ in the parentheses to get
\begin{multline}
	M_s = -\frac{|U_s|^2}{\nu_s|1+U_s|^2} + \frac{1}{|1+U_s|^2} 
	\Bigg[ \frac{\pi}{2 k_\parallel \sqrt{\pi} v_{th,s,\parallel}}
	\sum_n \exp(-k_\perp^2\bar{\rho}^2) I_n(k_\perp^2\bar{\rho}^2)\exp\big( -[y_n^*]^2\big) 	\\
	* \bigg( \exp \bigg[ \frac{i\nu_s (\omega-n\Omega_{cs})}{k_\parallel^2v_{th,s,\parallel}^2} \bigg] 
	\bigg[ - \underbrace{ i \erfi(y_n)}_{\erf(iy_n)} - i^2 \bigg] + \underbrace{i \erfi(y_n^*)}_{\erf(iy_n^*)} - i^2 \bigg)
	\Bigg]
\end{multline}
Simplify the $i$ terms and use Eq.~\ref{eq:a_erfiyn} to get
\begin{multline}
	M_s = -\frac{|U_s|^2}{\nu_s|1+U_s|^2} + \frac{1}{|1+U_s|^2} 
	\Bigg[ \frac{\sqrt{\pi}}{2 k_\parallel  v_{th,s,\parallel}}
	\sum_n \exp(-k_\perp^2\bar{\rho}^2) I_n(k_\perp^2\bar{\rho}^2)\exp\big( -[y_n^*]^2\big) 	\\
	* \bigg( \exp \bigg[ \frac{i\nu_s (\omega-n\Omega_{cs})}{k_\parallel^2v_{th,s,\parallel}^2} \bigg] 
	\bigg[\underbrace{1 - \erf(iy_n)}_{\erfc(i y_n)}  \bigg] +\erf(iy_n^*) + 1 \bigg)
	\Bigg].
\end{multline}
Use Eq.~\ref{eq:a_erfc} to get
\begin{multline}
	M_s = -\frac{|U_s|^2}{\nu_s|1+U_s|^2} + \frac{1}{|1+U_s|^2} 
	\Bigg[ \frac{\sqrt{\pi}}{2 k_\parallel v_{th,s,\parallel}}
	\sum_n \exp(-k_\perp^2\bar{\rho}^2) I_n(k_\perp^2\bar{\rho}^2)\exp\big( -[y_n^*]^2\big) 	\\
	* \bigg( \exp \bigg[ \frac{i\nu_s (\omega-n\Omega_{cs})}{k_\parallel^2v_{th,s,\parallel}^2} \bigg] 
	\erfc(i y_n)+\erf(iy_n^*) + 1 \bigg)
	\Bigg],
\end{multline}
which is the final nice form.  

REDO ALL OF THIS MODIFIED DISTRIBUTION STUFF. IT'S NOT QUITE RIGHT.

Rewrite all the final equations here. Then eventually show how this result simplifies to the fully Maxwellian case in the limit where $v_{th,s,\perp}=v_{th,s,\parallel}$.	

\section{Pole Integrals for a Maxwellian}
\label{a:poleIntegrals_Maxwellian}
This section provides the simplifications for the analytical solutions for the pole integrations of a normalized Maxwellian.
Initial analytical solutions are obtained from the Mathematica notebook \verb*|exact_poleIntegrate.nb|.

Consider a Maxwellian distribution of the form
\begin{equation}
	f = \exp(-v^2).
	\label{eq:a_maxwellian}
\end{equation}

For a singular pole at $z$, the exact solution is
\begin{equation}
	p_1(z) = \int_{-\infty}^\infty \frac{f}{v-z} dv = 
	\exp(-z^2) \Bigg[ -\pi \erfi(z) + \ln\bigg(\frac{1}{z}\bigg) + \ln(z)   \Bigg].
\end{equation}
We will call this solution function $p_1(z)$ since it will be used to define the solutions to the more complicated pole integrals.

For a set of poles at $z$ and its complex conjugate $z^*$, the exact solution is
\begin{align}
	p_*(z) &=  \int_{-\infty}^\infty \frac{f}{(v-z)(v-z^*)} dv  \nonumber \\
	&= \frac{i}{\im(z)} \Bigg\{ \underbrace{\exp(-z^2) \Bigg[ \pi \erfi(z) - \ln\bigg(\frac{1}{z}\bigg) - \ln(z)  \Bigg]}_{-p_1(z)}
	+ \underbrace{\exp\big(-[z^*]^2\big) \Bigg[ -\pi \erfi(z^*) + \ln \bigg(\frac{1}{z^*} + \ln(z^*) \bigg) \Bigg]}_{p_1(z^*)}
	\Bigg\}	\nonumber \\
	&= \frac{i}{2\im(z)} \bigg[ p_1(z^*) - p_1(z) \bigg].
\end{align}
As can be seen, the result can be written based on the function $p_1(z)$.

Lastly, for a second order pole at $z$, the solution is
\begin{align}
	p_2(z) &= \int_{-\infty}^\infty \frac{f}{(v-z)^2} dv  \nonumber \\
	&= -2 \sqrt{\pi} - 2\underbrace{ \exp(-z^2) \Bigg[ -\pi \erfi(z) + \ln\bigg(\frac{1}{z}\bigg) + \ln(z)  \Bigg]}_{p_1(z)} \nonumber \\
	&= -2 \sqrt{\pi} - 2 p_1(z).
\end{align}
