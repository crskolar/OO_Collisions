\chapter{Pole Integration Testing}

Put all of my tests here. 

\section{Testing functions used within poleIntegrate}

% Redo these tests, these are all now wrong that we changed the bounds of the outer part of the mesh refinement
\subsection{test\_getVInterp}
For \verb|test_getVInterp|, we have two poles at $2+i$ and $3-2i$ 
and an input velocity mesh of ranging from 0 to 10 in increments of 0.1.
Based on this, we can manually calculate that the expected velocity mesh for a \verb|mesh_n| of 11 to be
[0, 0.4, 0.6, 0.8, 1, 1.2, 1.4, 1.6, 2, 2.2, 2.4, 2.8, 3, 3.2, 3.6, 3.8, 4, 4.6, 4.8, 5, 5.4, 5.6, 6, 6.2, 6.4, 7, 7.2, 8, 8.6, 8.8, 9, 9.6, 10]. 
This has been tested and the function works as intended.

\subsection{test\_getIntegrand}
For \verb|test_getIntegrand|, we are trying to calculate 
\begin{equation}
	\frac{f_0(v)}{\prod_j (v-p_j)^{\mathcal{O}_j}},
	\label{eq:a_integrand}
\end{equation}
where $f_0(v)$ is the distribution function and $p_j$ are the poles with orders $\mathcal{O}_j$.
For an input of velocity of [0,1,2], an input distribution function of [2,3,1]
and two poles at $2+i$ and $3-2i$ with orders 1 and 2 respectively. 
Plugging these values into Eq.~\ref{eq:a_integrand} and using WolframAlpha, we get that the integrand should be
[$-0.05207-0.04497i$ ,$-0.1875-0.1875i$, $-0.16-0.12i$].
This has been tested and the function works as intended.

\subsection{test\_plemelj}

The exact solution for the integral over the real axis is provided by the Plemelj theorem, % Make a note on section whatever for where we talk about this more.
and is 
\begin{equation}
	\int_{-\infty}^\infty \frac{f_0(v)}{\prod_j (v-p_j)^{\mathcal{O}_j}} = 
	i \pi \sum_j \frac{ \sgn(\im[p_j]) f(\re[p_j])}{ \prod_{j\neq k} (p_j - p_k)^{\mathcal{O}_j}}.
	\label{eq:a_plemelj}
\end{equation} 
This solution is technically for the limit as the poles approach the real axis. 
Nevertheless, we can use it to test the numerical integration.
For a velocity mesh of [1, 2, 3], 
poles of $2-i$, $3+2i$, and $1-5i$ with orders 
1, 2, and 1, respectively, the exact solution using Eq.~\ref{eq:a_plemelj} (and WolframAlpha) is
\begin{equation}
	\int_{-\infty}^\infty \frac{f_0(v)}{\prod_j (v-p_j)^{\mathcal{O}_j}} = 
	i \pi \bigg(-\frac{11+7i}{170} \verb|Fv[1]| + \frac{96+247i}{140450} \verb|Fv[2]|  + \frac{26+15i}{901} \verb|Fv[0]|  \bigg) 
\end{equation}
This has been tested and the function works as intended.


\section{Testing poleIntegrate}

To test the pole integration, we will use a Guassian for the distribution function,
\begin{equation}
	f_0(v) = \exp(-v^2). 
	\label{eq:norm_var_maxwellian}
\end{equation}
One can think of this as examining a Maxwellian distribution with a velocity variable normalized by the thermal velocity.
For obtaining the incoherent scatter spectra, we will need to use this pole integration for integrands 
with singular poles at some $a+\gamma i$ as well as integrands with double poles at $a\pm \gamma i$.
%In addition, the Plemelj theorem holds true in the limit as $\gamma$ approaches zero. 
Therefore, we will make plots of the integrals as functions of $\gamma$ to show this.
For our case, we will have $\gamma$ vary and let $a=0.25$. 
To relate this back to Eqs. (REF HERE), the imaginary component of the poles is $-\nu/k_\parallel$.
For a proper comparison with the normalized variable Maxwellian in Eq.~\ref{eq:norm_var_maxwellian},
we will let $\gamma = -(\nu/k_\parallel)/v_{th}$. 
Therefore, the following plots can provide guidance on the velocity resolution needed to obtain accurate results.		
Please see the Mathematica notebook \verb*|exact_poleIntegrate.nb| for the analytic solutions to these integrals. % Reword

For a singular pole at $z$, the exact solution is
\begin{equation}
	p_1(z) = \int_{-\infty}^\infty \frac{f}{v-z} dv = 
	\exp(-z^2) \Bigg[ -\pi \erfi(z) + \ln \Big(-\frac{1}{z}\Big) + \ln\Big(\frac{1}{z}\Big)\Bigg]\\
	\label{eq:exact_single_pole}
\end{equation}
For utility later with the other solutions, we will call this solution $p_1(z)$.

%\begin{figure}[!htb]
%	\centering
%	\includegraphics[width=\linewidth]{p1.pdf}
%	\caption{Comparisons for a single pole integral calculation *Eq.~\ref{eq:exact_single_pole})
%		between the pole refined mesh integration function we made (blue circles), % Eventually say that better 
%		a naive trapezoidal integration not considering effects of poles (orange dots), and
%		the exact solution described by Eq.~\ref{eq:exact_single_pole} (black line)
%		for varying $\Delta v$.}
%	\label{f:single_pole_comparison}
%\end{figure}

Fig.~\ref{f:single_pole_comparison} shows how our pole integration compares with 
a naive trapezoidal integration and the exact solution from Eq.~\ref{eq:exact_single_pole},
as the pole approaches the real axis.
The approximate calculation uses a refined mesh of 101 points. 
It was found that additional points did not affect the solution in an appreciable way regardless of choice of $\Delta v$.
For large $\Delta v$, our pole integration clearly provides a better result
compared to the naive trapezoidal integration. 
As the $\Delta v$ decreases, both our approximation and the naive calculation improve.
For sufficiently small $\Delta v$, within the range of $\gamma$ chosen, both our
approximation and the naive solution converge.


For a double pole at $v - z$ and $v-z^*$ (where $z^*$ is the complex conjugate of $z$), the analytical solution to the integral is
\begin{equation}
	\int_{-\infty}^\infty \frac{\exp (-v^2)}{(v-z)(v-z^*)} dv = 
	\frac{i \Big[ g(z^*) - g(z)  \Big]}{2 \im(z)},
	\label{eq:exact_double_pole}
\end{equation}
where $g(z)$ is the solution to the single pole integrand (Eq.~\ref{eq:exact_single_pole}).


Fig.~\ref{f:double_pole_comparison} shows the comparisons to the exact solution for varying $\Delta v$.
Here, we find that the finer mesh pole integration, for all $\Delta v$, correctly calculates that the 
imaginary component of the solution is zero.
The naive trapezoidal integration gets better with lower $\Delta v$, but still does not reach anything close to zero.
For the real part of the solution, however, the finer mesh approximation we use does worse 
than the naive integration. While hte naive integration is still incorrect, it is less incorrect than the pole refined mesh integration.
Thus suggests for these double poles, we want to be the pole to be further from the real axis to obtain an accurate solution
with the pole refined mesh integration. 
Otherwise, perhaps the solution is to use the pole refined mesh integration for the imaginary component and the naive trapezoidal integration for the real component.

%\begin{figure}[!htb]
%	\centering
%	\begin{subfigure}{.32\textwidth}
%		\includegraphics[width=\linewidth]{Gaussian_vs_Plemelj_changeGamma_doublePole_1e0}
%		\caption{$\Delta v = 10^0$}
%	\end{subfigure}
%	\begin{subfigure}{.32\textwidth}
%		\includegraphics[width=\linewidth]{Gaussian_vs_Plemelj_changeGamma_doublePole_1e-1}
%		\caption{$\Delta v = 10^{-1}$}
%	\end{subfigure}
%	\begin{subfigure}{.32\textwidth}
%		\includegraphics[width=\linewidth]{Gaussian_vs_Plemelj_changeGamma_doublePole_1e-2}
%		\caption{$\Delta v = 10^{-2}$}
%	\end{subfigure}
%	\begin{subfigure}{.32\textwidth}
%		\includegraphics[width=\linewidth]{Gaussian_vs_Plemelj_changeGamma_doublePole_1e-3}
%		\caption{$\Delta v = 10^{-3}$}
%	\end{subfigure}
%	\begin{subfigure}{.32\textwidth}
%		\includegraphics[width=\linewidth]{Gaussian_vs_Plemelj_changeGamma_doublePole_1e-4}
%		\caption{$\Delta v = 10^{-4}$}
%	\end{subfigure}
%	\begin{subfigure}{.32\textwidth}
%		\includegraphics[width=\linewidth]{Gaussian_vs_Plemelj_changeGamma_doublePole_1e-5}
%		\caption{$\Delta v = 10^{-5}$}
%	\end{subfigure}
%	\caption{Comparisons for a double pole integral calculation
%		between the pole refined mesh integration function we made (blue solid line), % Eventually say that better 
%		a naive trapezoidal integration not considering effects of poles (orange dots), 
%		the exact solution described by Eq.~\ref{eq:exact_double_pole} (red dashed lined), 
%		and the Plemelj limit described by Eq.~\ref{eq:a_plemelj} (black dashed line)
%		for varying $\Delta v$.}
%	\label{f:double_pole_comparison}
%\end{figure}

For a single pole at $z$ with order 2, the analytical solution to the integral is. 
% Do this test at some point
\begin{equation}
	\int_{-\infty}^\infty \frac{ \exp(-v^2) }{(v-z)^2} dv = - 2 \sqrt{\pi} - 2 z g(z).
\end{equation}


\section{Testing spectra calculations}
The following are tests to ensure the $U$, $M$, and $\chi$ are being calculated properly.
Then, if those are correct, we can properly calculate the resulting ISR spectra for a particular ion distribution function.






