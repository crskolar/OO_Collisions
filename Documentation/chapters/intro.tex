\chapter{Introduction}

\section{O+O Stuff}




\section{ISR Theory}
\label{s:ISR-spectra}
The scattering spectra, $S$, of an electromagnetic wave on a one ion species magnetized collisional plasma is
\begin{equation}
	S(\omega,\mathbf{k}) = 2 \Big| 1 - \frac{\chi_e}{\epsilon}\Big|^2 M_e
	+ 2\Big|\frac{\chi_e}{\epsilon}\Big|^2 M_i,
	\label{eq:scattering-spectra}
\end{equation}
where $\chi$ is the susceptibility, 
$M$ is the modified distribution function,
and $\epsilon$ is the dielectric function defined as
\begin{equation}
	\epsilon = 1 + \chi_e + \chi_i.
	\label{eq:dielectric}
\end{equation}
To obtain these values, we first need to calculate the collisional term, $U_s$. % Include something on the collisional terms
For a species $s$ with normalized distribution function $f_{0s}(v)$, the 
\begin{equation}
	U_s = i \nu_s \sum_n \int 
	\frac{J_n^2\Big( \tfrac{k_\perp v_\perp}{\Omega_{cs}} \Big)}
	{\omega - k_\parallel v_\parallel - n\Omega_{cs} - i\nu_s}  
	f_{0s}(v)   d\mathbf{v}^3 ,
	\label{eq:Us}
\end{equation}
where $\nu$ is the total collision frequency,
$\Omega_{c}$ is the gyrofrequency,
$J_n$ is the $n$-th order Bessel function of the first kind,
and the integrals are taken over all velocity space.
The susceptibility is
\begin{equation}
	\chi_s = \frac{\omega_p^2}{k^2(1+U_s)} \sum_n \int 
	\frac{J_n^2\Big( \tfrac{k_\perp v_\perp}{\Omega_{cs}} \Big)}
	{\omega - k_\parallel v_\parallel - n\Omega_{cs} - i\nu_s}  
	\mathbf{k} \cdot \frac{\partial f_{0s}}{\partial \mathbf{v}}   d\mathbf{v}^3 ,
	\label{eq:chis}
\end{equation} 
where $\omega_p$ is the plasma frequency
and the derivative term is defined as
\begin{equation}
	\mathbf{k} \cdot \frac{\partial f_{0s}}{\partial \mathbf{v}} = 
	k_\parallel \frac{\partial f_{0s}}{\partial v_\parallel}
	+ \frac{n \Omega_{cs}}{v_\perp} \frac{\partial f_{0s}}{\partial v_\perp}.
	\label{eq:chis_derivative}
\end{equation}
The modified distribution function is 
\begin{equation}
	M_s = \frac{\nu_s}{|1+U_s|^2}
	\Bigg( - \frac{|U_s|^2}{\nu_s^2} 
	+ \sum_n \int 
	\frac{J_n^2\Big( \tfrac{k_\perp v_\perp}{\Omega_{cs}} \Big)}
	{(\omega - k_\parallel v_\parallel - n\Omega_{cs})^2 + \nu_s^2}
	f_{0s}(v)   d\mathbf{v}^3 \Bigg).
	\label{eq:Ms}
\end{equation}

Due to Bragg scattering, we must note that the effective wavenumber is two times that of what you would calculate for the ISR transmit frequency.
For example, the standard approach would be to calculate the radar wavelength as $\lambda=c/\nu$ where $c$ is the speed of light and $\nu$ is the radar frequency.
Then the wavenumber is related to the wavelength as $k=2\pi/\lambda$. 
But, due to the constructive interference from Bragg scattering, we need to multiply this wavenumber by two giving us 
\begin{equation}
	k = \frac{4\pi \nu}{c}.
	\label{eq:k}
\end{equation}


For a normalized Maxwellian distribution,
\begin{equation}
	f_{0s} = 
	\frac{1}{v_{th,s}^3\pi^{3/2}}
	\exp\bigg[- \frac{(v_\perp^2 + v_\parallel^2)}{v_{th,s}^2} \bigg],
\end{equation}
the collisional term, susceptibility, and modified distribution function are
\begin{equation}
	U_s = 
	\frac{i \nu_{s}}{k_\parallel v_{th,s}} 
	\sum_n \exp(-k_\perp^2 \bar{\rho}_s^2) I_n (k_\perp^2 \bar{\rho}_s^2) 
	\Big[ 2 \Da(y_n) + i \sqrt{\pi} \exp(-y_n^2) \Big]
	\label{eq:Us_exact}
\end{equation}
\begin{equation}
	\chi_s = 
	\frac{\alpha^2}{1+U_s} \frac{T_e}{T_s} 
	\sum_n \exp(-k_\perp^2 \bar{\rho}_s^2) I_n (k_\perp^2 \bar{\rho}_s^2) 
	\bigg[ 1 - \frac{\omega-i\nu_s}{k_\parallel v_{th,s}} 
	\Big( 2 \Da[y_n] + i \sqrt{\pi} \exp[-y_n^2] \Big)
	\bigg]
	\label{eq:chis_exact}
\end{equation}
\begin{equation}
	M_s = 
	\Bigg[
	\frac{1}{|1+U_s|^2} \frac{\sqrt{\pi}}{k_\parallel v_{th,s}}
	\sum_n \exp(-k_\perp^2 \bar{\rho}_s^2) I_n (k_\perp^2 \bar{\rho}_s^2) \exp(-y_n^2)
	\Bigg]
	- \frac{|U_s|^2}{\nu_s |1+U_s|^2},
\end{equation}
where
\begin{equation}
	y_n = \frac{\omega-n\Omega_{cs}-i\nu_s}{k_\parallel v_{th,s}}
\end{equation}
\begin{equation}
	\bar{\rho}_s = \frac{v_{th,s}}{\sqrt{2}\Omega_{cs}}
\end{equation}
\begin{equation}
	\alpha = \frac{1}{k \lambda_{D_e}}.
\end{equation}
See Appendix~\ref{a:spectra} for the full derivation.


